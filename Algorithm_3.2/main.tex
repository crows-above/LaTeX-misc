\documentclass{article}
\usepackage{amsmath}

\title{Introduction to Algorithms: Section 3.2 exercises}
\author{crow}
\date{February 2025}

\begin{document}

\maketitle

\section{3.2-1}
\quad Let $f(n)$ and $g(n)$ be asymptotically nonnegative functions. Using the basic definition of $\Theta$-notation, prove that $max\{f(n),g(n)\} \in$ $\Theta$$(f(n) + g(n))$ \newline

Solution: 
\begin{equation}
    max\{f(n),g(n)\} =
        \begin{cases}
            f(n), & \text{if } f(n) > g(n)\\
            g(n), & \text{otherwise}
        \end{cases}
\end{equation}
\quad $max\{f(n),g(n)\}$ grows at least as fast as $f(n)$, as if $f(n)$ is returned, it is larger than $g(n)$. Thus $max\{f(n),g(n)\} \in$ $\Omega$$(f(n))$. And because $f(n)$ and $g(n)$ are both nonnegative, $f(n) + g(n)$ is larger than $f(n)$ or $g(n)$ individually. So it can be said that $max\{f(n),g(n)\} \in O(f(n) + g(n))$, because $max\{f(n),g(n)\}$ can grow no faster than $f(n) + g(n)$. And because $\Theta$-notation characterizes the rate of growth from a constant factor above, and a constant factor below. It can be said that $max\{f(n),g(n)\} \in$ $\Theta$$(f(n) + g(n)$, given $O(f(n) + g(n))$ and $\Omega$$(f(n))$

\section{3.2-2}
\quad Explain why the statement, "The running time of algorithm $A$ is at least $O(n$\textsuperscript{2}$)$," is meaningless

\section{3.2-3}
\quad Is $2\textsuperscript{n+1} \in O(2\textsuperscript{n})?$ Is $2\textsuperscript{2n} \in O(2\textsuperscript{n})?$

\section{3.2-4}
\quad Prove Theorem 3.1: \newline
For any two functions $f(n)$ and $g(n)$, we have $f(n) \in \Theta(g(n))$ if and only if $f(n) \in O(g(n))$ and $f(n) \in \Omega(g(n))$

\section{3.2-5}
\quad Prove that the running time of an algorithm is $\Theta(g(n))$ if and only if its worst-case running time is $O(g(n))$ and its best-case running time is $\Omega(g(n))$

\section{3.2-6}
\quad Prove that $o(g(n)) \cap \omega(g(n))$ is the empty set

\section{3.2-7}
\quad We can extend our notation to the case of two parameters $n$ and $m$ that can go to $\infty$ independently at different rates. For a given function $g(n,m)$, we denote by $O(g(n,m))$ the set of functions \newline

$O(g(n,m)) \in \{f(n,m) :$ there exist positive constants $c, n\textsubscript{0}$, and $m\textsubscript{0}$ such that $0\leq f(n,m)\leq cg(n,m)$ for all $n\geq n\textsubscript{0}$ or $m\geq m\textsubscript{0}\}$ \newline

Give corresponding definitions for $\Omega(g(n,m))$ and $\Theta(g(n,m

\section{Additional Notes}
\quad This is still a work in progress
\end{document}
